%%
%% Copyright 2007, 2008, 2009 Elsevier Ltd
%%
%% This file is part of the 'Elsarticle Bundle'.
%% ---------------------------------------------
%%
%% It may be distributed under the conditions of the LaTeX Project Public
%% License, either version 1.2 of this license or (at your option) any
%% later version.  The latest version of this license is in
%%    http://www.latex-project.org/lppl.txt
%% and version 1.2 or later is part of all distributions of LaTeX
%% version 1999/12/01 or later.
%%
%% The list of all files belonging to the 'Elsarticle Bundle' is
%% given in the file `manifest.txt'.
%%

%% Template article for Elsevier's document class `elsarticle'
%% with numbered style bibliographic references
%% SP 2008/03/01
%%
%%
%%
%% $Id: elsarticle-template-num.tex 4 2009-10-24 08:22:58Z rishi $
%%
%%
%%\documentclass[preprint,12pt,3p]{elsarticle}
\documentclass[12pt,5p]{elsarticle}

%% Use the option review to obtain double line spacing
%% \documentclass[preprint,review,12pt]{elsarticle}

%% Use the options 1p,twocolumn; 3p; 3p,twocolumn; 5p; or 5p,twocolumn
%% for a journal layout:
%% \documentclass[final,1p,times]{elsarticle}
%% \documentclass[final,1p,times,twocolumn]{elsarticle}
%% \documentclass[final,3p,times]{elsarticle}
%% \documentclass[final,3p,times,twocolumn]{elsarticle}
%% \documentclass[final,5p,times]{elsarticle}
%%\documentclass[final,5p,times,twocolumn]{elsarticle}


%% if you use PostScript figures in your article
%% use the graphics package for simple commands
%% \usepackage{graphics}
%% or use the graphicx package for more complicated commands
%% \usepackage{graphicx}
%% or use the epsfig package if you prefer to use the old commands
%%\usepackage{epsfig}

%% The amssymb package provides various useful mathematical symbols
\usepackage{amssymb}
%% The amsthm package provides extended theorem environments
%% \usepackage{amsthm}

%% The lineno packages adds line numbers. Start line numbering with
%% \begin{linenumbers}, end it with \end{linenumbers}. Or switch it on
%% for the whole article with \linenumbers after \end{frontmatter}.
%%\usepackage{lineno}

%% natbib.sty is loaded by default. However, natbib options can be
%% provided with \biboptions{...} command. Following options are
%% valid:

%%   round  -  round parentheses are used (default)
%%   square -  square brackets are used   [option]
%%   curly  -  curly braces are used      {option}
%%   angle  -  angle brackets are used    <option>
%%   semicolon  -  multiple citations separated by semi-colon
%%   colon  - same as semicolon, an earlier confusion
%%   comma  -  separated by comma
%%   numbers-  selects numerical citations
%%   super  -  numerical citations as superscripts
%%   sort   -  sorts multiple citations according to order in ref. list
%%   sort&compress   -  like sort, but also compresses numerical citations
%%   compress - compresses without sorting
%%
%% \biboptions{comma,round}


% \biboptions{}

\journal{Computers, Environment and Urban Systems}
%%\journal{Nuclear Physics B}

\begin{document}

\begin{frontmatter}

\title{Proposta de Infraestrutura de Dados Espaciais para o Setor El\'etrico no Brasil} %%\tnoteref{label0}%%
%%\title{Sample article to present \texttt{elsarticle} class\tnoteref{label0}}
%%\tnotetext[label0]{Programa de P\’os Gradua\c{c}\~ao em Geografia}
%%\tnotetext[label0]{This is only an example}

\author{D. S. Candido\corref{cor1}}
%%\author[label1,label2]{Author One\corref{cor1}\fnref{label3}}
\address{Brasilia, Brasil}
%\address[label1]{Address One}
%\address[label2]{Address Two\fnref{label4}}

\cortext[cor1]{Endere\c{c}o: SGAN 603 Brasilia Brasil Cep: 70830-110}
%\cortext[cor1]{I am corresponding author}
%\fntext[label3]{I also want to inform about\ldots}
%\fntext[label4]{Small city}

\ead{diogosantana@aneel.gov.br}
%\%ead{author.one@mail.com}
\ead[url]{sigel.aneel.gov.br}
%%\ead[url]{author-one-homepage.com}

%\author[label5]{Author Two}
%\address[label5]{Some University}
%\ead{author.two@mail.com}

%\author[label1,label5]{Author Three}
%\ead{author.three@mail.com}

\begin{abstract}
Text of abstract. 
\end{abstract}


\begin{keyword}
%% keywords here, in the form: keyword \sep keyword
Palavras chaves
%%example \sep \LaTeX \sep template
%% MSC codes here, in the form: \MSC code \sep code
%% or \MSC[2008] code \sep code (2000 is the default)
\end{keyword}

\end{frontmatter}

%%
%% Start line numbering here if you want
%%
% \linenumbers

%% main text
\section{Introdu\c{c}\~ao}
\label{sec1}

Sample text. . Citation of Einstein paper~\cite{Einstein}.
%Sample text. . Citation of Einstein paper~\cite{e}.

%\subsection{Sample subsection}
%\label{subsec1}

%Sample text. 

%% The Appendices part is started with the command \appendix;
%% appendix sections are then done as normal sections

\section{Infraestrutura Nacional de Dados Espaciais no Brasil}

Sample text

\section{Infraestrutura de Dados Espaciais no Setor El\'etrico}

Sample text

\section{Metodologia para implantar uma Infraestrutura de Dados Espaciais}

Sample text

\section{Resultados}

Sample text

\section{Discuss\~ao}

Sample text

\section{Conclus\~ao}

Sample text

\appendix

\section{Descri\c{c}\~ao dos atributos}
\label{appendix-sec1}

Sample text. 

%% References
%%
%% Following citation commands can be used in the body text:
%% Usage of \cite is as follows:
%%   \cite{key}         ==>>  [#]
%%   \cite[chap. 2]{key} ==>> [#, chap. 2]
%%

%% References with bibTeX database:

%\bibliographystyle{elsarticle-num}
% \bibliographystyle{elsarticle-harv}
% \bibliographystyle{elsarticle-num-names}
% \bibliographystyle{model1a-num-names}
% \bibliographystyle{model1b-num-names}
% \bibliographystyle{model1c-num-names}
% \bibliographystyle{model1-num-names}
% \bibliographystyle{model2-names}
% \bibliographystyle{model3a-num-names}
% \bibliographystyle{model3-num-names}
% \bibliographystyle{model4-names}
\bibliographystyle{model5-names}
% \bibliographystyle{model6-num-names}

\bibliography{sample}


\end{document}

%%
%% End of file `elsarticle-template-num.tex'.
